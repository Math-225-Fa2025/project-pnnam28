\documentclass[]{tufte-handout}

% ams
\usepackage{amssymb,amsmath}

\usepackage{ifxetex,ifluatex}
\usepackage{fixltx2e} % provides \textsubscript
\ifnum 0\ifxetex 1\fi\ifluatex 1\fi=0 % if pdftex
  \usepackage[T1]{fontenc}
  \usepackage[utf8]{inputenc}
\else % if luatex or xelatex
  \makeatletter
  \@ifpackageloaded{fontspec}{}{\usepackage{fontspec}}
  \makeatother
  \defaultfontfeatures{Ligatures=TeX,Scale=MatchLowercase}
  \makeatletter
  \@ifpackageloaded{soul}{
     \renewcommand\allcapsspacing[1]{{\addfontfeature{LetterSpace=15}#1}}
     \renewcommand\smallcapsspacing[1]{{\addfontfeature{LetterSpace=10}#1}}
   }{}
  \makeatother

\fi

% graphix
\usepackage{graphicx}
\setkeys{Gin}{width=\linewidth,totalheight=\textheight,keepaspectratio}

% booktabs
\usepackage{booktabs}

% url
\usepackage{url}

% hyperref
\usepackage{hyperref}

% units.
\usepackage{units}


\setcounter{secnumdepth}{-1}

% citations
\usepackage{natbib}
\bibliographystyle{plainnat}


% pandoc syntax highlighting
\usepackage{color}
\usepackage{fancyvrb}
\newcommand{\VerbBar}{|}
\newcommand{\VERB}{\Verb[commandchars=\\\{\}]}
\DefineVerbatimEnvironment{Highlighting}{Verbatim}{commandchars=\\\{\}}
% Add ',fontsize=\small' for more characters per line
\newenvironment{Shaded}{}{}
\newcommand{\AlertTok}[1]{\textcolor[rgb]{1.00,0.00,0.00}{\textbf{#1}}}
\newcommand{\AnnotationTok}[1]{\textcolor[rgb]{0.38,0.63,0.69}{\textbf{\textit{#1}}}}
\newcommand{\AttributeTok}[1]{\textcolor[rgb]{0.49,0.56,0.16}{#1}}
\newcommand{\BaseNTok}[1]{\textcolor[rgb]{0.25,0.63,0.44}{#1}}
\newcommand{\BuiltInTok}[1]{\textcolor[rgb]{0.00,0.50,0.00}{#1}}
\newcommand{\CharTok}[1]{\textcolor[rgb]{0.25,0.44,0.63}{#1}}
\newcommand{\CommentTok}[1]{\textcolor[rgb]{0.38,0.63,0.69}{\textit{#1}}}
\newcommand{\CommentVarTok}[1]{\textcolor[rgb]{0.38,0.63,0.69}{\textbf{\textit{#1}}}}
\newcommand{\ConstantTok}[1]{\textcolor[rgb]{0.53,0.00,0.00}{#1}}
\newcommand{\ControlFlowTok}[1]{\textcolor[rgb]{0.00,0.44,0.13}{\textbf{#1}}}
\newcommand{\DataTypeTok}[1]{\textcolor[rgb]{0.56,0.13,0.00}{#1}}
\newcommand{\DecValTok}[1]{\textcolor[rgb]{0.25,0.63,0.44}{#1}}
\newcommand{\DocumentationTok}[1]{\textcolor[rgb]{0.73,0.13,0.13}{\textit{#1}}}
\newcommand{\ErrorTok}[1]{\textcolor[rgb]{1.00,0.00,0.00}{\textbf{#1}}}
\newcommand{\ExtensionTok}[1]{#1}
\newcommand{\FloatTok}[1]{\textcolor[rgb]{0.25,0.63,0.44}{#1}}
\newcommand{\FunctionTok}[1]{\textcolor[rgb]{0.02,0.16,0.49}{#1}}
\newcommand{\ImportTok}[1]{\textcolor[rgb]{0.00,0.50,0.00}{\textbf{#1}}}
\newcommand{\InformationTok}[1]{\textcolor[rgb]{0.38,0.63,0.69}{\textbf{\textit{#1}}}}
\newcommand{\KeywordTok}[1]{\textcolor[rgb]{0.00,0.44,0.13}{\textbf{#1}}}
\newcommand{\NormalTok}[1]{#1}
\newcommand{\OperatorTok}[1]{\textcolor[rgb]{0.40,0.40,0.40}{#1}}
\newcommand{\OtherTok}[1]{\textcolor[rgb]{0.00,0.44,0.13}{#1}}
\newcommand{\PreprocessorTok}[1]{\textcolor[rgb]{0.74,0.48,0.00}{#1}}
\newcommand{\RegionMarkerTok}[1]{#1}
\newcommand{\SpecialCharTok}[1]{\textcolor[rgb]{0.25,0.44,0.63}{#1}}
\newcommand{\SpecialStringTok}[1]{\textcolor[rgb]{0.73,0.40,0.53}{#1}}
\newcommand{\StringTok}[1]{\textcolor[rgb]{0.25,0.44,0.63}{#1}}
\newcommand{\VariableTok}[1]{\textcolor[rgb]{0.10,0.09,0.49}{#1}}
\newcommand{\VerbatimStringTok}[1]{\textcolor[rgb]{0.25,0.44,0.63}{#1}}
\newcommand{\WarningTok}[1]{\textcolor[rgb]{0.38,0.63,0.69}{\textbf{\textit{#1}}}}

% table with pandoc

% multiplecol
\usepackage{multicol}

% strikeout
\usepackage[normalem]{ulem}

% morefloats
\usepackage{morefloats}


% tightlist macro required by pandoc >= 1.14
\providecommand{\tightlist}{%
  \setlength{\itemsep}{0pt}\setlength{\parskip}{0pt}}

% title / author / date
\title{Final Project}
\author{Nam Pham}
\date{}


\begin{document}

\maketitle




\section{Research Question: What lifestyle and health factors most
affect sleep
quality?}\label{research-question-what-lifestyle-and-health-factors-most-affect-sleep-quality}

\section{1. Data Setup}\label{data-setup}

Here I import my dataset in sleep and set it to clean\_names for a
further data cleaning.

\begin{Shaded}
\begin{Highlighting}[]
\NormalTok{sleep }\OtherTok{\textless{}{-}} \FunctionTok{read.csv}\NormalTok{(}\StringTok{"data/Sleep\_health\_and\_lifestyle\_dataset.csv"}\NormalTok{) }\SpecialCharTok{\%\textgreater{}\%}
  \FunctionTok{clean\_names}\NormalTok{()}
\end{Highlighting}
\end{Shaded}

\begin{verbatim}
## [1] 374  13
\end{verbatim}

\begin{verbatim}
## Rows: 374
## Columns: 13
## $ person_id               <int> 1, 2, 3, 4, 5, 6, 7, 8, 9, 10, 11, 12, 13, 14,~
## $ gender                  <chr> "Male", "Male", "Male", "Male", "Male", "Male"~
## $ age                     <int> 27, 28, 28, 28, 28, 28, 29, 29, 29, 29, 29, 29~
## $ occupation              <chr> "Software Engineer", "Doctor", "Doctor", "Sale~
## $ sleep_duration          <dbl> 6.1, 6.2, 6.2, 5.9, 5.9, 5.9, 6.3, 7.8, 7.8, 7~
## $ quality_of_sleep        <int> 6, 6, 6, 4, 4, 4, 6, 7, 7, 7, 6, 7, 6, 6, 6, 6~
## $ physical_activity_level <int> 42, 60, 60, 30, 30, 30, 40, 75, 75, 75, 30, 75~
## $ stress_level            <int> 6, 8, 8, 8, 8, 8, 7, 6, 6, 6, 8, 6, 8, 8, 8, 8~
## $ bmi_category            <chr> "Overweight", "Normal", "Normal", "Obese", "Ob~
## $ blood_pressure          <chr> "126/83", "125/80", "125/80", "140/90", "140/9~
## $ heart_rate              <int> 77, 75, 75, 85, 85, 85, 82, 70, 70, 70, 70, 70~
## $ daily_steps             <int> 4200, 10000, 10000, 3000, 3000, 3000, 3500, 80~
## $ sleep_disorder          <chr> "None", "None", "None", "Sleep Apnea", "Sleep ~
\end{verbatim}

\section{2. Data Cleaning}\label{data-cleaning}

In the dataset, blood pressure is a string with / symbols so I want to
remove it by using separate to remove it and set the blood pressure to
be an integer (Numeric).Also set the category varibles in dataset to be
factors.

\begin{Shaded}
\begin{Highlighting}[]
\CommentTok{\# clean dataset }
\NormalTok{sleep\_clean }\OtherTok{\textless{}{-}}\NormalTok{ sleep }\SpecialCharTok{\%\textgreater{}\%}
  \FunctionTok{separate}\NormalTok{(blood\_pressure, }\AttributeTok{into =} \FunctionTok{c}\NormalTok{(}\StringTok{"systolic"}\NormalTok{, }
                                    \StringTok{"diastolic"}\NormalTok{), }
           \AttributeTok{sep=} \StringTok{"/"}\NormalTok{, }\AttributeTok{remove =} \ConstantTok{TRUE}\NormalTok{) }\SpecialCharTok{\%\textgreater{}\%}
  \FunctionTok{mutate}\NormalTok{(}
    \AttributeTok{systolic  =} \FunctionTok{as.numeric}\NormalTok{(systolic),}
    \AttributeTok{diastolic =} \FunctionTok{as.numeric}\NormalTok{(diastolic)}
\NormalTok{  )}

\CommentTok{\#factor the catagories variables}
\NormalTok{sleep\_clean }\OtherTok{\textless{}{-}}\NormalTok{ sleep\_clean }\SpecialCharTok{\%\textgreater{}\%}
  \FunctionTok{mutate}\NormalTok{(}
    \AttributeTok{gender        =} \FunctionTok{factor}\NormalTok{(gender),}
    \AttributeTok{bmi\_category  =} \FunctionTok{factor}\NormalTok{(bmi\_category, }
                           \AttributeTok{levels =} \FunctionTok{c}\NormalTok{(}\StringTok{"Underweight"}\NormalTok{, }
                                      \StringTok{"Normal"}\NormalTok{, }
                                      \StringTok{"Overweight"}\NormalTok{, }
                                      \StringTok{"Obese"}\NormalTok{),}
                           \AttributeTok{ordered =} \ConstantTok{TRUE}\NormalTok{),}
    \AttributeTok{sleep\_disorder =} \FunctionTok{factor}\NormalTok{(sleep\_disorder),}
    \AttributeTok{occupation    =} \FunctionTok{factor}\NormalTok{(occupation)}
\NormalTok{  )}

\FunctionTok{glimpse}\NormalTok{(sleep\_clean)}
\end{Highlighting}
\end{Shaded}

\begin{verbatim}
## Rows: 374
## Columns: 14
## $ person_id               <int> 1, 2, 3, 4, 5, 6, 7, 8, 9, 10, 11, 12, 13, 14,~
## $ gender                  <fct> Male, Male, Male, Male, Male, Male, Male, Male~
## $ age                     <int> 27, 28, 28, 28, 28, 28, 29, 29, 29, 29, 29, 29~
## $ occupation              <fct> Software Engineer, Doctor, Doctor, Sales Repre~
## $ sleep_duration          <dbl> 6.1, 6.2, 6.2, 5.9, 5.9, 5.9, 6.3, 7.8, 7.8, 7~
## $ quality_of_sleep        <int> 6, 6, 6, 4, 4, 4, 6, 7, 7, 7, 6, 7, 6, 6, 6, 6~
## $ physical_activity_level <int> 42, 60, 60, 30, 30, 30, 40, 75, 75, 75, 30, 75~
## $ stress_level            <int> 6, 8, 8, 8, 8, 8, 7, 6, 6, 6, 8, 6, 8, 8, 8, 8~
## $ bmi_category            <ord> Overweight, Normal, Normal, Obese, Obese, Obes~
## $ systolic                <dbl> 126, 125, 125, 140, 140, 140, 140, 120, 120, 1~
## $ diastolic               <dbl> 83, 80, 80, 90, 90, 90, 90, 80, 80, 80, 80, 80~
## $ heart_rate              <int> 77, 75, 75, 85, 85, 85, 82, 70, 70, 70, 70, 70~
## $ daily_steps             <int> 4200, 10000, 10000, 3000, 3000, 3000, 3500, 80~
## $ sleep_disorder          <fct> None, None, None, Sleep Apnea, Sleep Apnea, In~
\end{verbatim}

\section{3. Data Summary}\label{data-summary}

\subsection{3.1 Summary of Key
Variables}\label{summary-of-key-variables}

Here I will compute some data summary to observe some of the
information.

\begin{verbatim}
##  sleep_duration  quality_of_sleep physical_activity_level  stress_level  
##  Min.   :5.800   Min.   :4.000    Min.   :30.00           Min.   :3.000  
##  1st Qu.:6.400   1st Qu.:6.000    1st Qu.:45.00           1st Qu.:4.000  
##  Median :7.200   Median :7.000    Median :60.00           Median :5.000  
##  Mean   :7.132   Mean   :7.313    Mean   :59.17           Mean   :5.385  
##  3rd Qu.:7.800   3rd Qu.:8.000    3rd Qu.:75.00           3rd Qu.:7.000  
##  Max.   :8.500   Max.   :9.000    Max.   :90.00           Max.   :8.000  
##    heart_rate     daily_steps         age       
##  Min.   :65.00   Min.   : 3000   Min.   :27.00  
##  1st Qu.:68.00   1st Qu.: 5600   1st Qu.:35.25  
##  Median :70.00   Median : 7000   Median :43.00  
##  Mean   :70.17   Mean   : 6817   Mean   :42.18  
##  3rd Qu.:72.00   3rd Qu.: 8000   3rd Qu.:50.00  
##  Max.   :86.00   Max.   :10000   Max.   :59.00
\end{verbatim}

\emph{The descriptive statistics show that participants in this dataset
sleep an average of 7.13 hours per night, with sleep durations ranging
from 5.8 to 8.5 hours. Self-reported sleep quality ranges from 4 to 9,
with a mean of 7.31, indicating generally good perceived sleep among the
sample. Lifestyle indicators vary widely: physical activity levels range
from 30 to 90 (mean ≈ 59), daily steps range from 3,000 to 10,000 (mean
≈ 6,817), and stress levels fall between 3 and 8 (mean ≈ 5.38),
suggesting a moderately stressed population. Health-related measures
show average heart rate around 70 bpm. Participant ages span from 27 to
59, with a mean age of 42.}

\subsection{3.2 Sleep Outcomes by
Disorder}\label{sleep-outcomes-by-disorder}

\begin{verbatim}
## # A tibble: 3 x 6
##   sleep_disorder     n mean_sleep_duration sd_sleep_duration mean_sleep_quality
##   <fct>          <int>               <dbl>             <dbl>              <dbl>
## 1 Insomnia          77                6.59             0.387               6.53
## 2 None             219                7.36             0.732               7.63
## 3 Sleep Apnea       78                7.03             0.975               7.21
## # i 1 more variable: sd_sleep_quality <dbl>
\end{verbatim}

\emph{Comparing outcomes across sleep disorder categories reveals clear
differences. Individuals with Insomnia experience the shortest sleep
duration (6.59 hours) and the lowest sleep quality (6.53), reflecting
the typical impact of insomnia on rest. Participants with no diagnosed
sleep disorder report the best overall sleep, averaging 7.36 hours and a
sleep quality of 7.63. The Sleep Apnea group falls between the other
two, with moderately reduced sleep duration (7.03 hours) and sleep
quality (7.21). These patterns suggest that sleep disorders meaningfully
affect both objective sleep duration and subjective sleep quality.}

\subsection{3.3 Visual Explorations}\label{visual-explorations}

\begin{center}\includegraphics{Project_files/figure-latex/sleep_vs_disorder-1} \end{center}

\emph{The boxplot comparing sleep quality across sleep disorder
categories shows clear differences between groups. Participants with no
sleep disorder have the highest median sleep quality. The insomnia group
has a noticeably lower median and a narrower range, indicating
consistently poor sleep ratings. The sleep apnea group shows a median
between the other two groups but with greater spread, reflecting more
variability in quality. Overall, the visualization reinforces the
pattern that diagnosed sleep disorders are associated with lower sleep
quality.}

\begin{center}\includegraphics{Project_files/figure-latex/sleep_vs_stress-1} \end{center}

\emph{The scatterplot of sleep quality versus stress level displays a
negative trend, supported by the downward-sloping regression line. As
stress levels increase from 3 to 8, sleep quality generally declines.
While individual points show some variability, the pattern suggests a
clear inverse relationship between stress and perceived sleep quality.
This plot visually supports the hypothesis that higher stress is
associated with poorer sleep outcomes.}

\section{4. Data Linear Regression}\label{data-linear-regression}

\begin{verbatim}
## 
## Call:
## lm(formula = quality_of_sleep ~ sleep_duration + stress_level + 
##     physical_activity_level + sleep_disorder, data = sleep_clean)
## 
## Residuals:
##      Min       1Q   Median       3Q      Max 
## -1.47758 -0.11935  0.07315  0.19272  0.98661 
## 
## Coefficients:
##                            Estimate Std. Error t value Pr(>|t|)    
## (Intercept)                5.512515   0.425889  12.944  < 2e-16 ***
## sleep_duration             0.489498   0.049949   9.800  < 2e-16 ***
## stress_level              -0.413061   0.020678 -19.976  < 2e-16 ***
## physical_activity_level    0.004679   0.001127   4.151 4.12e-05 ***
## sleep_disorderNone         0.352408   0.058100   6.066 3.27e-09 ***
## sleep_disorderSleep Apnea  0.241136   0.070364   3.427 0.000679 ***
## ---
## Signif. codes:  0 '***' 0.001 '**' 0.01 '*' 0.05 '.' 0.1 ' ' 1
## 
## Residual standard error: 0.3908 on 368 degrees of freedom
## Multiple R-squared:  0.8948, Adjusted R-squared:  0.8934 
## F-statistic: 626.3 on 5 and 368 DF,  p-value: < 2.2e-16
\end{verbatim}

The regression model explains a very large portion of variation in sleep
quality (R² = 0.895), meaning that the included lifestyle and health
variables do an excellent job predicting how well people sleep. Let's
break down the coefficients:

\textbf{Sleep duration: +0.49}

For each additional hour of sleep, sleep quality increases by about 0.49
points

→ People who sleep more tend to have higher sleep quality.

\textbf{Stress level: --0.41}

For every 1-point increase in stress, sleep quality drops by 0.41
points.

→ Higher stress strongly lowers sleep quality.

\textbf{Physical activity: +0.0047}

Small coefficient but still statistically significant, every 1 point
changes will increase 0.0047 in sleep quality units.

→ Small positive effect.

\textbf{Sleep disorder:}

\emph{None vs Insomnia --- Positive Effect (β = +0.35)}

Participants with no sleep disorder score 0.35 points higher in sleep
quality than those with insomnia.

\emph{Sleep Apnea vs Insomnia --- Positive Effect (β = +0.24)}

Those with sleep apnea also report better sleep quality than the
insomnia group.

→ People without disorders have higher sleep quality.

\section{5. Data Summary With
Function}\label{data-summary-with-function}

\begin{verbatim}
## # A tibble: 4 x 4
##   bmi_category     n mean_quality sd_quality
##   <ord>        <int>        <dbl>      <dbl>
## 1 Normal         195         7.66      0.946
## 2 Overweight     148         6.90      1.25 
## 3 Obese           10         6.4       1.90 
## 4 <NA>            21         7.43      1.40
\end{verbatim}

\begin{verbatim}
## # A tibble: 2 x 4
##   gender     n mean_quality sd_quality
##   <fct>  <int>        <dbl>      <dbl>
## 1 Female   185         7.66      1.28 
## 2 Male     189         6.97      0.999
\end{verbatim}

\begin{center}\includegraphics{Project_files/figure-latex/unnamed-chunk-3-1} \end{center}

\begin{center}\includegraphics{Project_files/figure-latex/unnamed-chunk-4-1} \end{center}

\section{Final Conclusion}\label{final-conclusion}

This project examined how lifestyle and health factors predict adult
sleep quality using the Sleep Health and Lifestyle dataset. The results
from exploratory analysis and multiple linear regression consistently
showed that stress level and sleep duration are the strongest predictors
of sleep quality. Higher stress is strongly associated with worse sleep,
while longer sleep duration leads to better subjective sleep ratings.

Physical activity also plays a meaningful role, contributing positively
to sleep quality, although its effect is smaller compared to stress and
sleep duration. Sleep disorders, especially insomnia, significantly
reduce sleep quality, even after adjusting for lifestyle variables. This
suggests that both behavioral and medical factors influence sleep
outcomes. BMI category and demographic variables showed smaller but
noticeable differences in sleep quality.

Overall, the findings indicate that improving sleep quality may require
addressing both stress management and healthy sleep habits, while also
considering clinical sleep issues such as insomnia. Although the dataset
is observational and cannot establish causation, the results provide a
strong statistical foundation for understanding which lifestyle and
health factors matter most for good sleep.



\end{document}
